\documentclass[11pt,a4paper]{article}
\usepackage[shorthands=off,english]{babel} % package for multilingual support

\usepackage[utf8]{inputenc}
\usepackage[T1]{fontenc}
\usepackage{microtype}
\usepackage{csquotes}

\usepackage{xcolor}
\definecolor{dark-red}{rgb}{0.6,0.15,0.15}
\definecolor{dark-green}{rgb}{0.15,0.4,0.15}
\definecolor{medium-blue}{rgb}{0,0,0.5}
\usepackage[ pdfauthor={Vladimir Still}
           , pdftitle={Functional Aspects of Contemporary and Future C++},
           , pdfsubject={Presentation Abstract},
           , plainpages=false
           , pdfpagelabels
           , unicode
           , draft=false
           , colorlinks=true
           , linkcolor={dark-red}
           , citecolor={dark-green}
           , urlcolor={medium-blue}
           , unicode=true
           ]{hyperref}

\pagestyle{empty}

\begin{document}
\section*{Functional Aspects of Contemporary and Future C++}

\renewcommand{\abstractname}{Presentation Abstract}
\begin{abstract}
	\noindent
	Modern C++ is a multi-paradigm programming language which features aspects
	of procedural, object-oriented, generic, and functional programming.
	In this presentation, we will focus on its functional aspects and relate its
	features to some of the common functional patterns from functional
	languages such as Haskell.
	We will see that list transformation functions such as \texttt{map} and
	\texttt{filter} have their C++ counterparts in the newly standardised ranges
	library.
	We will also show how algebraic data types can be represented in C++, and
	that future of C++ might include features such as pattern matching and use
	of monads.
\end{abstract}

\end{document}
