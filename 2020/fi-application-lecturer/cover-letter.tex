\documentclass[11pt,a4paper]{article}
\usepackage[shorthands=off,english]{babel} % package for multilingual support

\usepackage[utf8]{inputenc}
\usepackage[T1]{fontenc}
\usepackage{microtype}
\usepackage{csquotes}

\usepackage{xcolor}
\definecolor{dark-red}{rgb}{0.6,0.15,0.15}
\definecolor{dark-green}{rgb}{0.15,0.4,0.15}
\definecolor{medium-blue}{rgb}{0,0,0.5}
\usepackage[ pdfauthor={Vladimir Still}
           , pdftitle={Application for the Lecturer in Programming Position},
           , pdfsubject={Conver Letter},
           , plainpages=false
           , pdfpagelabels
           , unicode
           , draft=false
           , colorlinks=true
           , linkcolor={dark-red}
           , citecolor={dark-green}
           , urlcolor={medium-blue}
           , unicode=true
           ]{hyperref}

\usepackage{amssymb,amsmath,amsthm}
\usepackage{enumitem}

\begin{document}
Dear members of the selection committee,

\smallskip
I would like to apply for the position of Lecturer in Programming.
I like to teach, both in seminars and in lectures and I also like to find and implement ways to improve courses.
My first experience with teaching was in 2011 when I started teaching seminars for the Introduction to Functional Programming course on our faculty.
Since then, I kept some involvement with teaching each year, either as a seminar tutor or as a person responsible for the creation and grading of homework.
I also have some experience with giving lectures as I was substitute lecturer for a few lectures of C++ Programming, Advanced Programming in C++, and Non-Imperative Programming.
Furthermore, I am one of the founders of the Advanced Programming in C++ course on our faculty, and I helped re-design both the Non-Imperative Programming course and Seminar on Functional Programming.
Over the years, I was involved in teaching Non-Imperative Programming, C, C++, and formal languages.
The list of my involvement in courses can be found in my CV.

Apart from the direct teaching experience, I have also experience with auto-grading systems.
In 2014 I created an auto-grading system for Haskell which allows students to submit their solutions through questionnaires in IS MU.
Since then I maintain this system and extend its capabilities, both for Haskell evaluation (used in Non-Imperative Programming course since 2014) and for evaluation of other languages using language-specific modules.
Currently, this system is used in Non-Imperative Programming, Foundations of Programming, Algorithms and Data Structures I, and Seminar on Functional Programming courses.
I am now also consulting a student who is writing a master's thesis, which allows the use of this auto-grader system for some types of homework in the Formal Languages and Automata course.
Furthermore, I have an accepted poster proposal to the conference on Innovation and Technology in Computer Science Education (ITiCSE) about the Haskell auto-grader.

My programming experience includes advanced knowledge of C++, good knowledge of Python, Haskell, and C and also some knowledge of C\#, Perl, and Prolog.
I believe that knowledge of multiple programming languages with different paradigms and relations between them is an essential part of teaching programming, even if the person would be teaching only one of them.
For this reason, I try to keep my skills in multiple programming languages.
I also follow closely the new developments in programming languages, mainly the new features of modern C++ and Haskell.

Apart from programming, I have also experience with more theoretical aspects of computer science and computer science research.
My PhD thesis is on the topic of verification of parallel C++ programs, and I am also interested in theories and practice around formal languages, computability, and complexity, and the theoretical foundations of programming languages.

I am used to working in a team, and I know from experience that it is better to prepare courses as a team and for larger courses this is a necessity.
I have experience of being both a regular team member and a leader of a subteam responsible for a certain part of the course.

As a lecturer, I would like to work on the programming courses, including Foundations of Programming, C++ Programming, Advanced Programming in C++, and Non-Imperative Programming.
I would also like to keep improving and using the auto-grader system I have designed and which is now used in many courses.

\smallskip
With kind regards,\par
Vladimír Štill
\end{document}
