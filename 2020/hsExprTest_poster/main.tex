% Conference: ITiCSE 2020, https://iticse.acm.org/
% Poster CfP: https://iticse.acm.org/posters/
% Deadline: 2020-03-08 AoE (i.e. 8:00 a.m. Monday 2020-03-09)

% Proceedings from the previous year for inspiration: https://dl.acm.org/doi/proceedings/10.1145/3304221
% For poster abstracts, see section "5C: Posters"

\documentclass[sigconf]{acmart} % Toggles: review, anonymous, authorversion, screen

\usepackage[utf8]{inputenc}         % Encoding of this TeX source file
\usepackage{csquotes}				% Proper quoting

\settopmatter{printacmref=true}
  % mandatory for ITiCSE'20

\fancyhead{}
  % do not delete this code.

\usepackage{balance}
  % for creating a balanced last page (usually last page with references)

% defining the \BibTeX command - from Oren Patashnik's original BibTeX documentation.
\def\BibTeX{{\rm B\kern-.05em{\sc i\kern-.025em b}\kern-.08emT\kern-.1667em\lower.7ex\hbox{E}\kern-.125emX}}

\copyrightyear{2020}
\acmYear{2020}
\setcopyright{rightsretained}
\acmConference[ITiCSE '20]{Proceedings of the 2020 ACM Conference on Innovation and Technology in Computer Science Education}{June 15--19, 2020}{Trondheim, Norway}
\acmBooktitle{Proceedings of the 2020 ACM Conference on Innovation and Technology in Computer Science Education (ITiCSE '20), June 15--19, 2020, Trondheim, Norway}
\acmPrice{15.00}
\acmDOI{10.1145/3341525.3393972}
\acmISBN{978-1-4503-6874-2/20/06}

\newcommand{\hsExprTest}{\textit{hsExprTest}}

\begin{document}

\fancyhead{}
  % do not delete this code.

\title{Automatic Test Generation\\ for Haskell Programming Assignments}

\author{Vladimír Štill}
% \orcid{}
\affiliation{
  \institution{Faculty of Informatics, Masaryk University}
  \country{Czech Republic}
}
\email{xstill@mail.muni.cz}

\begin{abstract}
    Automatic testing of programming assignments is highly desirable as it can
    provide fast feedback for the students and allows the teachers to teach
    efficiently even in courses with many students.
    However, writing tests for students' solutions can be tedious.
    In this work, we present a novel approach to test generation for small Haskell
    assignments.
    Such assignments usually consist of one function (with the possibility to use
    helper functions in its definition) that the students are supposed to program
    according to a teacher's specification.
    The teacher is not required to write tests for this function.
    Instead, we make use of an example solution, which the teacher should have
    to assess the difficulty of the assignment.
    Using the example solution, and (if needed) a specification of input values for
    the function, our tool can automatically generate randomized tests.
    If these tests fail, the student is presented with a counterexample which
    shows the input values, the expected output of the tested function and the
    output computed by their solution.
    % In this work, we present a tool which simplifies testing small programming
    % assignments written in the Haskell functional programming language.
    % We test these exercises at the level of functions -- given a solution to
    % the assignment (a function) and (if needed) specification of the input data
    % for the tested function, the tools can automatically derive tests, compare
    % the teacher's solution to the student's solution, and report the result.
\end{abstract}

\begin{CCSXML}
<ccs2012>
    <concept>
        <concept_id>10003456.10003457.10003527</concept_id>
        <concept_desc>Social and professional topics~Computing education</concept_desc>
        <concept_significance>500</concept_significance>
    </concept>
</ccs2012>
\end{CCSXML}

\ccsdesc[500]{Social and professional topics~Computing education}

\keywords{%
programming education,
automatic testing,
Haskell,
QuickCheck
}

\maketitle

\section{Testing Capabilities}

Our tool \hsExprTest{} is based on QuickCheck~\cite{Koen2000}, a Haskell tool for
specification-based testing of Haskell programs.
It uses QuickCheck for automatic generation of test inputs, and to shrink test
inputs, which allows us to produce small counterexamples.
It also uses QuickCheck's ability to generate printable and shrinkable
functions~\cite{Koen2012}, which is used when testing implementations of
higher-order functions (i.e., functions which take functions as arguments, for
example, a \texttt{map} function, which takes a function and applies it to each
element of a list, producing a list of results).

However, unlike QuickCheck, \hsExprTest{} uses the teacher's solution as a
specification and automatically derives a test expression which compares the
teacher's and student's solutions.
The expression generation is driven by the type of teacher's solution (and
optionally also input specification).
It can include specialization of
polymorphic functions and wrapping of inputs for higher-order functions in such
a way they can be displayed and shrunk by QuickCheck.

For the simplest exercise functions (total functions operating on basic
Haskell data types like numbers, lists, tuples, and functions over these
types), our tool can generate tests just from the solution.
For more complex input data, it is necessary to either constrain inputs
(for example to positive numbers, or to a specific range) or to provide means
of generation of random inputs.
Input data generators for custom data types can be written using the QuickCheck
library.
Finally, to accommodate more complex exercises within the same framework, it is
also possible to write custom QuickCheck tests.

\section{Usage and Availability}

Our tool is open source and available on GitHub, at
\url{https://github.com/vlstill/hsExprTest}.
The documentation for testing Haskell can be found at
\url{https://github.com/vlstill/hsExprTest/tree/master/doc/hs.md}.
The tool is used in the introductory course to functional programming at the
Faculty of Informatics of Masaryk University since 2014 and is continually
being improved.
While the tool is integrated with the Information System of Masaryk University,
which is used to submit assignments and present results to students, it is also
possible to use it with different submission frontends.

\section{Goals of the Poster}

The poster aims to present our novel testing method which simplifies testing of
small Haskell programming exercises.
We believe that by making testing easier, we can promote the creation of more
and better exercises which students can use to improve their new skills.
The poster could also spark a broader conversation about methods used in
programming courses.

\balance
\bibliographystyle{ACM-Reference-Format}
\bibliography{main}

\end{document}
