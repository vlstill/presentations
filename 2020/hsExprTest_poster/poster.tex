%%%%%%%%%%%%%%%%%%%%%%%%%%%%%%%%%%%%%%%%%
% Jacobs Portrait Poster
% LaTeX Template
% Version 1.0 (31/08/2015)
% (Based on Version 1.0 (29/03/13) of the landscape template
%
% Created by:
% Computational Physics and Biophysics Group, Jacobs University
% https://teamwork.jacobs-university.de:8443/confluence/display/CoPandBiG/LaTeX+Poster
%
% Further modified by:
% Nathaniel Johnston (nathaniel@njohnston.ca)
%
% Portrait version by:
% John Hammersley
%
% The landscape version of this template was downloaded from:
% http://www.LaTeXTemplates.com
%
% License:
% CC BY-NC-SA 3.0 (http://creativecommons.org/licenses/by-nc-sa/3.0/)
%
%%%%%%%%%%%%%%%%%%%%%%%%%%%%%%%%%%%%%%%%%

%----------------------------------------------------------------------------------------
%    PACKAGES AND OTHER DOCUMENT CONFIGURATIONS
%----------------------------------------------------------------------------------------

\documentclass[final, 20pt, a0]{beamer}


\usepackage[T1]{fontenc}
\usepackage[utf8]{inputenc}

\usepackage[scale=1.15]{beamerposter} % Use the beamerposter package for laying out the poster

\usepackage[ backend=biber
           , style=numeric
           , sortlocale=en_US
           , bibencoding=UTF8
           , sorting=anyt % explicit label, name, year, title
           , maxnames=4
           , maxalphanames=4
           , maxbibnames=100
           , urldate=long
           ]{biblatex}
\DeclareSourcemap{
    \maps[datatype=bibtex, overwrite]{
        \map{
            \step[fieldset=editor, null]
            \step[fieldset=language, null]
        }
    }
}
\addbibresource{main.bib}

\usetheme{confposter} % Use the confposter theme supplied with this template

\definecolor{mublue}{RGB}{27,69,143} % {cmyk}{0.81,0.52,0,0.44}
\setbeamercolor{structure}{fg=black}
\setbeamercolor{cboxb}{bg=mublue}
\setbeamercolor{title in headline}{fg=mublue}
\setbeamercolor{block title}{fg=mublue,bg=white} % Colors of the block titles
\setbeamercolor{block body}{fg=black,bg=white} % Colors of the body of blocks
\setbeamercolor{block alerted title}{fg=white,bg=mublue} % Colors of the highlighted block titles
\setbeamercolor{block alerted body}{fg=black,bg=mublue} % Colors of the body of highlighted blocks
% Many more colors are available for use in beamerthemeconfposter.sty

%-----------------------------------------------------------
% Define the column widths and overall poster size
% To set effective sepwid, onecolwid and twocolwid values, first choose how many columns you want and how much separation you want between columns
% In this template, the separation width chosen is 0.024 of the paper width and a 4-column layout
% onecolwid should therefore be (1-(# of columns+1)*sepwid)/# of columns e.g. (1-(4+1)*0.024)/4 = 0.22
% Set twocolwid to be (2*onecolwid)+sepwid = 0.464
% Set threecolwid to be (3*onecolwid)+2*sepwid = 0.708

\newlength{\sepwid}
\newlength{\marginwid}
\newlength{\onecolwid}
\newlength{\twocolwid}
\newlength{\threecolwid}
% \setlength{\paperwidth}{841mm} % A0 width: 46.8in
% \setlength{\paperheight}{1189mm} % A0 height: 33.1in
\setlength{\sepwid}{20mm} % Separation width (white space) between columns
\setlength{\marginwid}{80mm} % Margin width (white space) left and right
\setlength{\onecolwid}{214mm} % Width of one column
\setlength{\twocolwid}{\onecolwid} % Width of two columns
\addtolength{\twocolwid}{\onecolwid}
\addtolength{\twocolwid}{\sepwid}
\setlength{\threecolwid}{0.708\paperwidth} % Width of three columns
\setlength{\topmargin}{1cm}

\usepackage{xspace}
\usepackage{pdfpages}
\usepackage{microtype}
\usepackage{booktabs}
\usepackage[export]{adjustbox}

\newcommand{\TODO}[1]{\textcolor{red}{#1}}
\setlength{\overfullrule}{15pt} % TODO: remove

%-----------------------------------------------------------
\usepackage{wrapfig}
\usepackage{tikz}
\usetikzlibrary{arrows,arrows.meta,positioning,shapes,automata,backgrounds,petri,fit,matrix,decorations.pathreplacing,calc}
\tikzset{
    *|/.style={
        to path={
            (perpendicular cs: horizontal line through={(\tikztostart)},
                                 vertical line through={(\tikztotarget)})
            % is the same as (\tikztostart -| \tikztotarget)
            % but just to be safe: http://tex.stackexchange.com/a/29781/16595
            -- (\tikztotarget) \tikztonodes
        }
    },
	|*/.style={to path={
		(\tikztostart) -- (perpendicular cs: vertical line through={(\tikztostart)},
											 horizontal line through={(\tikztotarget)})
  	}}
}

\usepackage{graphicx}  % Required for including images
\usepackage{booktabs} % Top and bottom rules for tables
\usepackage[english]{babel}

\usepackage{varwidth}
\usepackage{minted}
\usepackage{qrcode}

\newminted[hscode]{haskell}{autogobble,breaklines}
\newmintinline[hs]{haskell}{}

\newcommand{\hsExprTest}{\textit{hsExprTest}}

%----------------------------------------------------------------------------------------
%    TITLE SECTION
%----------------------------------------------------------------------------------------

\newcommand{\titletext}{Automatic Test Generation\\ for Haskell Programming Assignments}
\newcommand{\logowidth}{2.6\baselineskip}
\title{%
  \texorpdfstring{%
    \makebox[\linewidth]{%
      \noindent\hspace*{0.5em} %
      \hfill %
      \begin{minipage}{\linewidth-3em}\center\titletext\end{minipage} %
      \hfill %
      \hspace*{0.5em}%
    }\bigskip%
  }
  {\titletext}}

\author{Vladimír Štill}

\institute{Faculty of Informatics, Masaryk University\\Czech Republic\vspace*{-2cm}} % Institution(s)

%----------------------------------------------------------------------------------------

\begin{document}

\addtobeamertemplate{block end}{}{\vspace*{2ex}} % White space under blocks
\addtobeamertemplate{block alerted end}{}{\vspace*{2ex}} % White space under highlighted (alert) blocks
\setbeamercolor{itemize item}{fg=mublue}

\setlength{\belowcaptionskip}{2ex} % White space under figures
\setlength\belowdisplayshortskip{2ex} % White space under equations

\begin{frame}[t] % The whole poster is enclosed in one beamer frame

\begin{columns}[t] % The whole poster consists of three major columns, the second of which is split into two columns twice - the [t] option aligns each column's content to the top

\begin{column}{\marginwid}\end{column} % Empty spacer column
\begin{column}{\onecolwid} % The first column

\begin{block}{Overview}
    Automatic testing of programming assignments is highly desirable as it can
    provide fast feedback for the students and allows the teachers to teach
    efficiently even in courses with many students.
    However, writing tests for students' solutions can be tedious.
    In this work, we present a novel approach to test generation for small Haskell
    assignments.
    Such assignments usually consist of one function (with the possibility to use
    helper functions in its definition) that the students are supposed to program
    according to a teacher's specification.
    The teacher is not required to write tests for this function.
    Instead, we make use of an example solution, which the teacher should have
    to assess the difficulty of the assignment.
    Using the example solution, and (if needed) a specification of input values for
    the function, our tool can automatically generate randomized tests.
    If these tests fail, the student is presented with a counterexample which
    shows the input values, the expected output of the tested function and the
    output computed by their solution.
\end{block}

\begin{block}{Our Use}
    In autumn 2019 we had roughly 950 students in the course, with 12 homework
    sets and 32 unique auto-evaluated exercises.
    Auto-evaluated exercises are also available for practice before exams.
\end{block}

\begin{block}{More Information}
    Our tool \hsExprTest{} is based on QuickCheck~\cite{Koen2000}, a tool for
    specification-based testing of Haskell programs.
    It is publicly available on GitHub, as part of a toolkit for automatic
    testing of small programming exercises used in several courses on the
    Faculty of Informatics of Masaryk University.

    \bigskip\bigskip
    More information and links to GitHub can be also found on the accompanying
    page for this poster.

    \bigskip\bigskip\footnotesize
    \qrcode[height=0.25\onecolwid]{http://vstill.eu/publications/2020/hsExprTest/}~~
    \href{http://vstill.eu/publications/2020/hsExprTest/}{vstill.eu/publications/2020/hsExprTest/}
\end{block}

\end{column} % End of the first column

\begin{column}{\sepwid}\end{column} % Empty spacer column

\begin{column}{\twocolwid} % Begin a column which is two columns wide (column 2)

\begin{block}{Tool Workflow}

    \begin{figure}[h]
        \begin{center}
        \begin{tikzpicture}[
                >=stealth', shorten >=1pt, auto, node distance=3cm, semithick,
                scale=2, thick,
                code/.style = {align = left, draw, inner sep = 0.5em,
                               rounded corners = 0.3em},
                ->/.style = {-{Latex[length=0.5em]}}
               ]

            \node[code] (student) {
                \hs{binmap :: (a -> a -> b) -> [a] -> [b]}\\
                \hs{binmap _ []       = []} \\
                \hs{binmap _ [_]      = []} \\
                \hs{binmap f (x:y:xs) = f x y : binmap f (y:xs)}
              };
            \node[above = 0em of student.north, anchor = south] {Student's Solution};

            \node[code, below = 3em of student.south west, anchor = north west] (teacher) {
                \hs{expr = "binmap"} \\
                \hs{timeout = 5} \\
                \\
                \hs{binmap :: (a -> a -> b) -> [a] -> [b]} \\
                \hs{binmap f xs = zipWith f xs (drop 1 xs)}
              };
            \node[above = 0em of teacher.north, anchor = south] {Teacher's Solution + Specification};

            \node[code, below right = 3em and -0.2em of teacher] (exprtest)
                {
                  \begin{varwidth}{14em}\begin{itemize}
                    \item compare types
                    \item instantiate polymorphic types
                    \item make functions showable
                    \item generate expression
                    \item run on random data
                  \end{itemize}\end{varwidth}
                };
            \node[above = 0em of exprtest.north, anchor = south] {\textbf{\hsExprTest}};

            \node[left = 3em of exprtest, yshift = 1.5em] (ok)
                {\color{green!50!black}\textbf{Test Passed}};
            \node[left = 3em of exprtest, yshift = -1.5em] (fail)
                {\color{red}\textbf{Test Failed + Counterexample}};

            \draw (student.east) -- ++(4.5em, 0) edge[|*, ->] (exprtest.north);
            \draw (teacher.east) -- ++(1.5em, 0) edge[|*, ->] (exprtest.north);
            \draw[->] (exprtest) edge (ok.east)
                      (exprtest) edge (fail.east);

        \end{tikzpicture}
        \end{center}

        \caption{%
            Workflow of our tool \hsExprTest{}.
            The student's code is submitted through the web browser usually, and
            the student can get their an answer within a few seconds.
        }
    \end{figure}

\end{block}

\begin{block}{Main Features}

\end{block}

\begin{block}{Bibliography}

{
    \emergencystretch 3em % there is little other fixing possible for bibio
    \printbibliography[heading=none]
}
\end{block}

\end{column} % End of the double

\begin{column}{\marginwid}\end{column} % Empty spacer column

\end{columns}

\end{frame}

\end{document}

% vim: spell spelllang=en tw=80 expandtab
