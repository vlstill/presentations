\documentclass[aspectratio=169, fi]{paradise-slide}

\setbeamertemplate{caption}[numbered]
\setbeamertemplate{caption label separator}{:}
\setbeamercolor{caption name}{fg=normal text.fg}
\usepackage{amssymb,amsmath}
\usepackage{ifxetex,ifluatex}
\usepackage{fixltx2e} % provides \textsubscript

\usepackage{upquote}
\usepackage{microtype}
\usepackage{booktabs}

\usetikzlibrary{arrows,shapes,snakes,automata,backgrounds,petri,fit}

% Comment these out if you don't want a slide with just the
% part/section/subsection/subsubsection title:
\AtBeginPart{
  \let\insertpartnumber\relax
  \let\partname\relax
  \frame{\partpage}
}
\AtBeginSection{
  \let\insertsectionnumber\relax
  \let\sectionname\relax
  \frame{\sectionpage}
}
\AtBeginSubsection{
  \let\insertsubsectionnumber\relax
  \let\subsectionname\relax
  \frame{\subsectionpage}
}

\setlength{\parindent}{0pt}
\setlength{\emergencystretch}{3em}  % prevent overfull lines
\providecommand{\tightlist}{%
  \setlength{\itemsep}{0pt}\setlength{\parskip}{0pt}}
\setcounter{secnumdepth}{0}
\ifxetex
  \usepackage{polyglossia}
  \setmainlanguage{}
  \setotherlanguages{}
\else
  \usepackage[shorthands=off,english]{babel}
\fi
\usepackage{minted}

\newminted[cppcode]{cpp}{autogobble,breaklines,escapeinside=⟦⟧}
\newminted[cppcodeln]{cpp}{autogobble,breaklines,linenos,escapeinside=⟦⟧}
\newmintinline[cpp]{cpp}{}

\usepackage[ backend=biber
           , style=alphabetic % numeric
           , sortlocale=en_US
           , bibencoding=UTF8
           , sorting=anyt % explicit label, name, year, title
           , sortlocale=cs_CZ
           , maxnames=100
           , maxalphanames=4
           , maxbibnames=100
           , urldate=long
           ]{biblatex}
\DeclareSourcemap{
    \maps[datatype=bibtex, overwrite]{
        \map{
            \step[fieldset=editor, null]
            \step[fieldset=language, null]
            \step[fieldset=url, null]
            \step[fieldset=isbn, null]
        }
    }
}
\addbibresource{my.bib}
\newcommand{\fcite}[1]{\emergencystretch 3em{\protect\NoHyper\cite{#1}}~\fullcite{#1}}

\usepackage{xspace}
\newcommand{\xtso}{\mbox{x86-\kern-.15em TSO}\xspace}

\title{Analysis of Parallel C++ Programs}
\subtitle{PhD Thesis Defense}
\author{Vladimír Štill}
\date{3rd December 2020}

\newenvironment{prespart}[1]{%
  \begin{frame}{}%
    \centering
      {\Large #1} \par\bigskip\bigskip%
}{%
  \end{frame}%
}
\newcommand{\rquote}[1]{\begin{quote}#1\end{quote}\bigskip\setlength{\leftmargini}{1em}}

\begin{document}

\frame[plain, noframenumbering]{\titlepage}

\begin{frame}{Analysis of Parallel C++ Programs}

\textbf{Automatic Analysis of Parallel Programs}
\begin{itemize}
    \item parallel processing to make use of modern hardware
    \item easy to make mistakes
    \item bugs hard to find by conventional testing
\end{itemize}
\pause

\bigskip
\textbf{C++}
\begin{itemize}
  \item performance-critical applications
  \item high-level language
  \item has many features present in other languages (and not present in C)
    \begin{itemize}
      \item complex objects, inheritance, threading, atomic variables, exceptions, …
    \end{itemize}
\end{itemize}
\end{frame}

\begin{frame}{DIVINE}
  All contributions presented in the thesis are implemented and evaluated in DIVINE

  \begin{itemize}
    \item explicit-state analyser for C and C++, using LLVM intermediate representation
      \begin{enumerate}
        \item C/C++ translated to LLVM using clang compiler
        \item LLVM is instrumented
        \item and executed by DIVINE in search for errors
      \end{enumerate}\pause
    \item primarily targets safety
      \begin{itemize}
        \item assertion safety, memory safety, arithmetic errors, …
        \item also supports Büchy and termination conditions
      \end{itemize} \pause
    \item symbolic extensions
    \item multiple other contributors, parts presented here authored mostly by me
    \item open source
  \end{itemize}
\end{frame}

\begin{prespart}{Improvements in Analysis of Realistic Programs}
  \begin{itemize}
    \item \fcite{SRB2017} \hfill 75\,\%
  \end{itemize}
\end{prespart}

\begin{frame}{Improvements in Analysis of Realistic Programs}
  analysis of high-level programming languages is complex \pause
  \begin{itemize}
    \item syntax, code generation,
    \item objects, run-time types, virtual methods
    \item exceptions
    \item extensive libraries
    \item …
    \item → expensive to implement from scratch
  \end{itemize}

  \pause\bigskip
  we reuse existing components
  \begin{itemize}
    \item compiler (clang)
    \item some libraries (pdclib, libc++)
  \end{itemize}

  and implement only what is necessary
  \begin{itemize}
    \item threads (pthread)
    \item parts of exception handling (libunwind)
  \end{itemize}
\end{frame}

\newcommand{\arr}[1]{\visible<#1>{\textcolor{red}{$\mathbf{\longleftarrow}$}}}
\begin{frame}[fragile]{Exceptions in C++}
  \begin{minipage}{0.47\textwidth}
    \setlength{\leftmargini}{1em}
    \begin{itemize}
      \item ubiquitous in C++, its standard library
      \item need complex runtime support
        \begin{itemize}
          \item exception matching
          \item cleanup
          \item stack unwinding
        \end{itemize}
    \end{itemize}

    \bigskip
    \visible<2->{Stack:}\medskip\\
    \visible<2,4>{\framebox[0.8\textwidth]{\vphantom{gf}\only<2>{\cpp{g}}\only<4>{\cpp{X::~X}}}}
    \visible<2-4>{\framebox[0.8\textwidth]{\vphantom{gf}\cpp{f}}}
    \visible<2->{\framebox[0.8\textwidth]{\vphantom{gf}\cpp{main}}}
  \end{minipage}
  \hfill
  \pause
  \begin{minipage}{0.47\textwidth}
    \begin{cppcodeln}
      X::~X() { /* ... */ } ⟦\arr{4}⟧

      void g() {
        throw std::out_of_range(); ⟦\arr{2}⟧
      }
      void f() {
        X x;
        g();
      } ⟦\arr{3}⟧
      int main() {
        try {
          f();
        } catch (std::logic_error&) {
          /* ... */ ⟦\arr{5}⟧
        }
      }
    \end{cppcodeln}
  \end{minipage}
\end{frame}

\begin{frame}{Exceptions in DIVINE}
  \textbf{Goals}
  \begin{itemize}
    \item reuse as much of the C++ library implementation as possible (libc++abi)
    \item limit changes to the core of DIVINE
    \item full support for C++ exceptions
  \end{itemize}
  \pause

  \bigskip
  \textbf{Our Approach}
  \begin{itemize}
    \item calculate exception handling metadata (for libc++abi and the unwinder)
    \item new unwinder using this metadata
    \item lower some exception handling primitives in LLVM to our functions
  \end{itemize}
  \pause

  \bigskip
  \textbf{Results}
  \begin{itemize}
    \item no modifications to libc++abi required
    \item performance comparable to a previous implementation that required substantial changes to
      the C++ library implementation and DIVINE
    \item makes use of the new state representation in DIVINE 4
  \end{itemize}
\end{frame}

\begin{prespart}{Analysis of Programs Under the \xtso Relaxed Memory Model}
  \begin{itemize}
    \item \fcite{SB2018x86tso} \hfill 90\,\%
  \end{itemize}
\end{prespart}

\begin{frame}[fragile]{Relaxed Memory}
  \begin{minipage}{0.47\textwidth}
    \begin{itemize}
      \item cache hierarchies
      \item instruction reordering
      \item speculative execution
    \end{itemize}

    … give rise to observable relaxed behaviour
  \end{minipage}
  \hfill\pause
  \begin{minipage}{0.48\textwidth}
    \begin{cppcode}
      int x = 0;
      int y = 0;
    \end{cppcode}
    \begin{minipage}{0.49\textwidth}
      \begin{cppcode}
        void thread0()
        { ⟦\arr{4}⟧
          y = 1; ⟦\arr{5}⟧
          int a = x; ⟦\arr{6-}⟧
        }
      \end{cppcode}
    \end{minipage}
    \hfill
    \begin{minipage}{0.49\textwidth}
      \begin{cppcode}
        void thread1()
        { ⟦\arr{4-6}⟧
          x = 1; ⟦\arr{7}⟧
          int b = y; ⟦\arr{8-}⟧
        }
      \end{cppcode}
    \end{minipage}

    \centering\medskip
    Is $\texttt{a} = 0 \land \texttt{b} = 0$ reachable?
  \end{minipage}

  \bigskip\bigskip\pause
  \begin{minipage}[t]{0.48\textwidth}
    Not under \emph{Sequential Consistency}

    \visible<4->{
      \medskip
      Globals: \texttt{x = \only<-6>{0}\only<7->{1}}, \texttt{y = \only<-4>{0}\only<5->{1}}\\
      Locals: \texttt{a = \only<-5>{$\bot$}\only<6->{0}}, \texttt{b = \only<-7>{$\bot$}\only<8->{1}}\\
    }
  \end{minipage}
  \hfill
  \begin{minipage}[t]{0.48\textwidth}
    Reachable on \textbf{x86}, ARM, POWER, …

    \visible<4->{
      \medskip
      Globals: \texttt{x = 0}, \texttt{y = 0}\\
      Locals: \texttt{a = \only<-5>{$\bot$}\only<6->{0}}, \texttt{b =
        \only<-7>{$\bot$}\only<8->{\textcolor{red}{0}}}\\
      Store buffer t0: \textcolor{red}{\only<5->{y = 1}} \\
      Store buffer t1: \textcolor{red}{\only<7-8>{x = 1}}  \\
    }
  \end{minipage}
\end{frame}

\begin{frame}[fragile]{Efficient explicit simulation of \xtso}
  \begin{itemize}
    \item \xtso – memory model of Intel and AMD x86(-64) CPUs
    \item transformation of LLVM programs
    \bigskip
    \item resolving order of buffered entries is expensive
    \item → \textbf{on-demand + delayed flushing}
  \end{itemize}

  \pause\bigskip
  \colorlet{frombuf}{green!60!black}
  \colorlet{flushed}{yellow!60!red!85!black}
  \begin{minipage}[t]{0.35\textwidth}

  \begin{tikzpicture}[ ->, >=stealth', shorten >=1pt, auto, node distance=3cm
                     , semithick
                     , scale=0.5
                     , thck/.style = { thick, decoration={markings,mark=at position 1 with {\arrow[scale=4]{>}}}, postaction={decorate}, },
                     ]

    \draw [-] (-10,0) rectangle (-7,-4);
    \draw [-] (-10,-1) -- (-7,-1)
              (-10,-2) -- (-7,-2)
              (-10,-3) -- (-7,-3);
    \node () [anchor=center] at (-8.5, 0.5) {s.b. 1};
    \node () [anchor=center] at (-8.5,-0.5) {\only<-5>{\only<4->{\color{flushed}}\texttt{x $\leftarrow$ 1}}};
    \node () [anchor=center] at (-8.5,-1.5) {\only<-3>{\texttt{y $\leftarrow$ 1}}\only<4-5>{\color{flushed}\texttt{x $\leftarrow$ 2}}};
    \node () [anchor=center] at (-8.5,-2.5) {\only<-3>{\texttt{x $\leftarrow$ 2}}};
    \node () [anchor=center] at (-8.5,-3.5) {\only<-3>{\only<3>{\color{frombuf}}\texttt{y $\leftarrow$ 2}}};

    \draw [-] (-6,0) rectangle (-3,-4);
    \draw [-] (-6,-1) -- (-3,-1)
              (-6,-2) -- (-3,-2)
              (-6,-3) -- (-3,-3);
    \node () [anchor=center] at (-4.5, 0.5) {s.b. 2};
    \node () [anchor=center] at (-4.5,-0.5) {\only<-6>{\only<5-6>{\color{frombuf}}\texttt{x $\leftarrow$ 3}}\only<7->{\texttt{y $\leftarrow$ 3}}};
    \node () [anchor=center] at (-4.5,-1.5) {\only<-6>{\texttt{y $\leftarrow$ 3}}};

  \end{tikzpicture}

  \newcommand{\ly}{\only<3->{\only<3>{\textcolor{frombuf}}{// \textrightarrow 2}}}
  \newcommand{\lx}{\only<5->{\only<5-6>{\textcolor{frombuf}}{// \textrightarrow 3}}}
  \begin{cppcode}
  void thread0() {
    int a = y; ⟦\ly⟧
    int b = x; ⟦\lx⟧
  }
  \end{cppcode}
  \end{minipage}
  \begin{minipage}[t]{0.60\textwidth}
      \begin{itemize}
          \only<1>{\item}
          \only<2>{\item three threads, thread 0 executing}
          \only<3>{\item nondeterministically choose which value to \emph{load} from the
          store buffers}
          \only<4>{\item older \texttt{y} values in s.b. 1 removed}
          \only<4>{\item other older values in s.b. 1 marked as flushed}
          \only<5>{\item nondeterministically choose to load from s.b.~2}
          \only<6>{\item flushed values for \texttt{x} has to be removed (overwritten)}
          \only<7>{\item \texttt{x} is propagated to the memory}
          \only<8>{\item \texttt{y $\leftarrow$ 3} stays in buffer}
      \end{itemize}
  \end{minipage}
\end{frame}

\begin{frame}{Results}

  \begin{itemize}
    \item evaluation on nonsymbolic SV-COMP concurrency benchmarks
    \item compared with CBMC (bounded) and Nidhugg (stateless)

    \vspace{2ex}
    \setlength\tabcolsep{0.5em} %
    \begin{tabular}{lrrr} \toprule
        \textcolor{gray}{(total = 54)} & CBMC & \textbf{DIVINE} & Nidhugg \\ \midrule
        finished                & 20   & 23     & 24 \\
        TSO errors              &  3   & 10     &  3 \\
        uniqly solved           &  5   &  6     &  8 \\
        \bottomrule
    \end{tabular}
  \end{itemize}
\end{frame}

\begin{prespart}{Local Nontermination analysis for Parallel Programs}
  \begin{itemize}
    \item \fcite{SB2019} \hfill 90\,\%
  \end{itemize}
\end{prespart}

\begin{frame}{Beyond Safety Checking}
  \begin{itemize}
    \item safety checking is common for realistic parallel programs
    \item if the program does not terminate…
    \begin{itemize}
      \item the tool might not terminate
      \item or it might correctly report there are no safety violations
    \end{itemize}
  \end{itemize}

  \bigskip
  → termination checking
\end{frame}

\begin{frame}[fragile]{What is Nontermination?}
  \begin{cppcode}
  std::atomic< bool > spin_lock;
  void worker() { // two of these running in parallel
      while (true) {
          while (spin_lock.exchange(true)) { /* wait */ }
          x++;
          spin_lock = false;
      }
  }
  \end{cppcode}

  Does every \emph{wait} end? \pause

  \bigskip
  \newcommand{\ta}[1]{\text{\textcolor{blue}{\texttt{[t0: #1]}}}}
  \newcommand{\tb}[1]{\text{\textcolor{red}{\texttt{[t1: #1]}}}}
  $\ta{spin\_lock.exchange(true) \textrightarrow{} false}$\\
  $\big(\tb{spin\_lock.exchange(true) \textrightarrow{} true}$\\
  $\phantom{\big(}\ta{x++}$\\
  $\phantom{\big(}\ta{spin\_lock = false}$\\
  $\phantom{\big(}\ta{spin\_lock.exchange(true) \textrightarrow{} false}\big)^\omega$

  \medskip
  \textbf{both threads can run}
\end{frame}

\begin{frame}[fragile]{Local Nontermination analysis for Parallel Programs}
  \setlength{\leftmargini}{1em}
  \begin{itemize}
    \item usable for (small) practical programs with arbitrary synchronization
    \item detection of nonterminating parts of programs
    \item no spurious nontermination reports
    \item simple specification
    \item bugs caused by unintended parallel interactions, not at complex control flow \& loops
  \pause
  \bigskip
    \item builds on explicit-state model checking\\
      → finite-state programs (with possibly infinite behaviour)
    \item user can specify what to check → resource sections
  \end{itemize}

%  \bigskip
%  \begin{minipage}{0.49\textwidth}
%    \begin{cppcode}
%      bool x = true;
%      while (true) { x = !x; }
%    \end{cppcode}
%  \end{minipage}
%  \hfill
%  \begin{minipage}{0.49\textwidth}
%    \begin{tikzpicture}[>=latex,>=stealth',auto,node distance=2cm,semithick,initial text=]
%      \node[state,initial] (t) {$x$};
%      \node[state] (f) [right of = t] {$\lnot x$};
%
%      \path[->, shorten >=1pt]
%        (t) edge[bend left] (f)
%        (f) edge[bend left] (t)
%      ;
%    \end{tikzpicture}
%  \end{minipage}
\end{frame}

\begin{frame}[fragile]{Local Nontermination analysis for Parallel Programs}
    \textbf{Resource Section}
    \begin{itemize}
      \item a block of code with an identifier
      \item delimited in the source code
      \item built-in or user-provided
    \end{itemize}
    \pause

    \bigskip
    \textbf{Local Nontermination}
    \begin{itemize}
      \item a resource section does not terminate if the program can reach apoint \emph{in the
        resource section} from which it cannot reach \emph{the corresponding resource section end}
    \end{itemize}
    \pause

    \bigskip
    \textbf{Detection}
    \begin{itemize}
      \item nondeterministically \emph{activate} resource sections at their beginning
      \item active resource sections are checked for nontermination, activation cannot be nested
      \item detect terminal strongly connected components in active resource sections
    \end{itemize}
\end{frame}

\begin{prespart}{Other Significant Publications}
  \begin{itemize}
    \item \fcite{DIVINEToolPaper2017} \hfill 30\,\%
    \bigskip
    \item \fcite{RSCB2018} \hfill 20\,\%
  \end{itemize}
\end{prespart}

\begin{prespart}{Conclusion}
  \begin{itemize}
    \item contributions to analysis of programs in C++
      \begin{itemize}
        \item realistic programs, exceptions in C++
          \\ – \emph{new approach to implementation of exceptions that reuses existing libraries}
          \medskip\pause
        \item programs running under the \xtso memory model
          \\ – \emph{new efficient way to simulate \xtso using an explicit-state model checker}
          \medskip\pause
        \item local nontermination detection
          \\ – \emph{new technique that can detect that a part of the program which is supposed to
          terminate does not}
          \medskip\pause
      \end{itemize}
    \item all published, implemented in DIVINE and evaluated
  \end{itemize}
\end{prespart}

\def\insertframenumber{A}
\begin{frame}[noframenumbering]{}
    \centering
      {\Large Reactions to Reviews}
\end{frame}

\def\rname{prof. Tomáš Vojnar}
\def\qtitle{Questions by \rname}

\begin{frame}[fragile, noframenumbering]{\qtitle}
\rquote{What I am slightly missing in the chapter [4] is a better explanation of which particular features
of exception handling were not supported by DIVINE 3.}

  \begin{itemize}
    \item exception specifications
      \begin{cppcode}
        void bar(int x) throw() { ... }
        void foo(int x) throw (std::out_of_range) { ... }
      \end{cppcode}
    \item new approach also allows for implementation of \cpp{longjmp}/\cpp{setjmp} for the sake of
      C~programs
    \item DIVINE 3 does not check if the target function is using C++ exceptions
  \end{itemize}
\end{frame}

\begin{frame}[noframenumbering]{\qtitle}
  Handling of unaligned loads in \xtso.\bigskip

  \begin{itemize}
    \item alignment is not significant for our \xtso simulation, as long as the there are no
      \emph{partially overlapping} loads/stores
      \begin{itemize}
        \item this matches with the observed behaviour of x86-64 processors and the \xtso
          theoretical memory model
      \end{itemize}
    \item partial overlaps should not happen for atomic variables in C++ (undefined behaviour)
    \item but can happen for non-atomics in some cases
    \bigskip
    \item partial overlaps not implemented completely
    \item the description should be extended to allow for loads from multiple store buffer entries
      if they partially overlap with loaded value
  \end{itemize}
\end{frame}

\begin{frame}[fragile, noframenumbering]{\qtitle}
  \rquote{Why do you concentrate solely on instances of resource sections that are bound to
    non-terminate? Why do you not consider those that can but need not terminate too? I understand
    that such non-termination can happen under weak fairness when accessing shared resources, but
    why not considering strong fairness (at least wrt access to some resources) and non-termation
    that is possible (though not necessary) under such fairness too?}

  Consider this example from a single-prodcer single-consumer queue using atomics:
  \begin{cppcode}
    T &front() {
        while ( empty() ) { /* wait */ }
        // ...
  \end{cppcode}
  \begin{itemize}
    \item \texttt{empty} is a non-blocking operation, therefore the thread is always enabled at this line
    \item we aimed to avoid false nontermination errors in common synchronization patterns
    \item resource asquisition cannot be easily discerned from the source code
    \item it would be possible to have also sections that check for other fairness criteria
  \end{itemize}
\end{frame}

\begin{frame}[noframenumbering]{\qtitle}
  \rquote{Where are the predefined resource sections applied in DIVINE? Is this within
    implementation of various synchronization primitives, or are you trying to detect even
    application-specific resource sections automatically?}

  \begin{itemize}
    \item in synchronization primitives defined in DIVINE (mostly in the \texttt{pthread} library)
    \item automatic detection might be possible for wait-style sections, but would be hard for
      mutual exclusion
    \item with automatic detection, there is a risk that a small change in source (or compiler)
      would interfere with detection
  \end{itemize}
\end{frame}

\def\rname{prof. Jaco van de Pol}

\begin{frame}[noframenumbering]{\qtitle}
  \rquote{Could you summarize the main problem(s) solved by this thesis, but \emph{not} in related
    work?}
  \begin{itemize}
    \item full support for C++ exception in a model checker
    \item most other contributions are improvements or manifest different strengths and weaknesses
      than related work
  \end{itemize}
\end{frame}

\begin{frame}[noframenumbering]{\qtitle}
  \rquote{Could you summarize the main new technique(s) developed by you in this thesis?}
  \begin{itemize}
    \item more efficient simulation of \xtso in the context of explicit-state model checking
    \medskip
    \item use of resource sections to denote interesting parts of parallel program and check that
      they terminate, togherther with the detection algorithm
  \end{itemize}
\end{frame}

\begin{frame}[noframenumbering]{\qtitle}
  \rquote{You state that the verification of nearly all features of C++ and its standard library are
    supported. Which features are not supported, why not, and how fundamental are they?}

  most of C++17 is supported, missing features include:
  \begin{itemize}
    \item some functionality related to locale support
%      \begin{itemize}
%        \item due to missing support in our libc implementation
%        \item unlikely to be used in parallel algorithms and data structures
%      \end{itemize}
      \pause

    \item filesystem support is limited by abilities of virtual file system in DIVINE
      \pause

    \item parts of \texttt{math.h/fenv.h} (mostly related to loating-point environment)
%      \begin{itemize}
%        \item missing support in the LLVM interpreter used in DIVINE
%        \item probably not much used
%      \end{itemize}
      \pause
    \item handling of real time
      \pause
    \item networking (defined in platform APIs, not C++)
  \end{itemize}
\end{frame}

\begin{frame}[noframenumbering]{\qtitle}
  \rquote{You transform an ``LLVM program under TSO'' to an ``LLVM program under SC''. What
  scientific support do you have that the transformation preserves semnatics? What properties of the
  program are actually preserved by the transformation? (e.g. all linerar-time properties? only
  stutter-free properties? fair traces? termination? branching properties?)}
  \begin{itemize}
    \item the argumentation stated in the paper and a suite of test which test reachable results of
      thread interactions \pause
    \item we considered mostly reachability
      \begin{itemize}
        \item automata-based approach to LTL would likely lead to infinite state space with store
          buffer limits, or unsound analysis without them
        \item as long as values used to evaluate the formula (or synchronize Büchy automaton, etc.)
          are read using the proposed transformation, unbounded state space should preserve
          stutter-free linear properties
      \end{itemize} \pause
    \item DIVINE can reliably analyse only stutter-free linerar-time properties and has no support
      for branching properties
  \end{itemize}
\end{frame}

\begin{frame}[noframenumbering]{\qtitle}
  \rquote{You compare your solution to weak-fairness, but how does your solution compare to strong
    fairness? Why are the fairness assumptions you postulate justified in reality?}
  \begin{itemize}
    \item strong fairness is not strong enough to eliminate spurious conterexamples in common
      synchronization patterns used with atomic variable
    \item preemptive schedulers use either priority-based fixed or varible (Linux) time-slices for
      scheduling, in either case the scheduler does not re-schedule after fixed number of
      \emph{instructions}
    \item therefore, if it is possible to leave a resource section \emph{by a choice of scheduler},
      it should eventually happen
    \item in part this is a compromise between reporting programs that can get stuck for long time,
      and reporting spurious errors
  \end{itemize}
\end{frame}

\begin{frame}[noframenumbering]{\qtitle}
  \rquote{Can it happen that the tool concludes termination, just because a malloc-function returns
    NULL (out-of-memory) non-deterministically?}
  \begin{itemize}
    \item yes (the problem is same as for symbolic data)
  \end{itemize}
\end{frame}

\begin{frame}[noframenumbering]{\qtitle}
  \rquote{Do you have concrete ideas on how to integrate your advances with more symbolic or more
    modular apporaches?}

  \textrm{I.} Symbolic approaches
  \begin{itemize}
    \item DIVINE has support for compiled data abstractions, both precise (bitvectors) and lossy
      \pause
    \item exception support does not interfere with symbolic approaches (as long as control-flow is
      explicit)
      \pause
    \item conceptually, \xtso simulation does not intefere with symbolic approaches, the program
      transformation does
      \begin{itemize}
        \item implementation would be tricky in DIVINE
          – the symbolic extension relies on over-approximation of set of possibly symbolic
          locations – this would likely encompass the whole program if any symbolic value was stored
          using store buffers
      \end{itemize}
      \pause
    \item local nontermination would need a modified fairness condition to handle nondeterminism
      without under-approximation
  \end{itemize}
\end{frame}

\begin{frame}[noframenumbering]{\qtitle}
  \rquote{Do you have concrete ideas on how to integrate your advances with more symbolic or more
    modular apporaches?}

  \textrm{II.} Modular approaches
  \begin{itemize}
    \item I have rather limited knowledge of modular approaches to software verification
      \pause
    \item exceptions
      \begin{itemize}
        \item symbolic analysis could be used to find conditions under which exception can
          or must be raised by a function
        \item however, our apporoach to exceptions is badly suited for axiomatic analysis usual for
          modular methods
      \end{itemize}
      \pause
    \item \xtso – I do not know about any modular approaches to relaxed memory anaylysis
      \pause
    \item local nontermination – thread-modular apporoaches seem to be orthogonal to our approach,
      it might be possible to use them to limit the number of resource sections that need to be
      checked
  \end{itemize}
\end{frame}

\begin{frame}[noframenumbering]{\qtitle}
  From text comments:
  \rquote{[Chapter 4] Is the solution sound with respect to the (intended) semantics of exceptions? In
    particular, what about corner cases, like an exception is raised during handling another
    exception (maybe even during stack unwinding/cleanup handlers).}

  yes
  \begin{itemize}
    \item these cases are handled by the part of C++ library that is re-used, so the behaviour
      shoudl be equivalent to the behaviour presented in normal execution
    \item only difference could be in parallel access to exception by the undwinder and some other
      thread, however:
      \begin{itemize}
        \item the exception object is thread-local
        \item while it contains user payload which might be globally visible, it does not access it
          (it~does not have any knowledge of its structure)
      \end{itemize}
  \end{itemize}
\end{frame}


\end{document}

% vim: fo+=t tw=100 spell spelllang=en shiftwidth=2
