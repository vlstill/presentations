\documentclass[aspectratio=169]{paradise-slide}

\setbeamertemplate{caption}[numbered]
\setbeamertemplate{caption label separator}{:}
\setbeamercolor{caption name}{fg=normal text.fg}
\usepackage{amssymb,amsmath}
\usepackage{ifxetex,ifluatex}
\usepackage{fixltx2e} % provides \textsubscript

\usepackage{upquote}
\usepackage{microtype}

% Comment these out if you don't want a slide with just the
% part/section/subsection/subsubsection title:
\AtBeginPart{
  \let\insertpartnumber\relax
  \let\partname\relax
  \frame{\partpage}
}
\AtBeginSection{
  \let\insertsectionnumber\relax
  \let\sectionname\relax
  \frame{\sectionpage}
}
\AtBeginSubsection{
  \let\insertsubsectionnumber\relax
  \let\subsectionname\relax
  \frame{\subsectionpage}
}

\setlength{\parindent}{0pt}
\setlength{\emergencystretch}{3em}  % prevent overfull lines
\providecommand{\tightlist}{%
  \setlength{\itemsep}{0pt}\setlength{\parskip}{0pt}}
\setcounter{secnumdepth}{0}
\ifxetex
  \usepackage{polyglossia}
  \setmainlanguage{}
  \setotherlanguages{}
\else
  \usepackage[shorthands=off,english]{babel}
\fi
\usepackage{minted}

\newminted[cppcode]{cpp}{autogobble,breaklines,escapeinside=⟦⟧}
\newminted[cppcodeln]{cpp}{autogobble,breaklines,linenos,escapeinside=⟦⟧}
\newmintinline[cpp]{cpp}{}

\usepackage[ backend=biber
           , style=alphabetic % numeric
           , sortlocale=en_US
           , bibencoding=UTF8
           , sorting=anyt % explicit label, name, year, title
           , sortlocale=cs_CZ
           , maxnames=100
           , maxalphanames=4
           , maxbibnames=100
           , urldate=long
           ]{biblatex}
\DeclareSourcemap{
    \maps[datatype=bibtex, overwrite]{
        \map{
            \step[fieldset=editor, null]
            \step[fieldset=language, null]
            \step[fieldset=url, null]
            \step[fieldset=isbn, null]
        }
    }
}
\addbibresource{my.bib}
\newcommand{\fcite}[1]{\emergencystretch 3em{\protect\NoHyper\cite{#1}}~\fullcite{#1}}

\usepackage{xspace}
\newcommand{\xtso}{\mbox{x86-\kern-.15em TSO}\xspace}

\title{Analysis of Parallel C++ Programs}
\subtitle{PhD Thesis Defense}
\author{Vladimír Štill}
\date{3rd December 2020}

\newenvironment{prespart}[1]{%
  \begin{frame}{}%
    \centering
      {\Large #1} \par\bigskip\bigskip%
}{%
  \end{frame}%
}
\let\otp\titlepage
\renewcommand{\titlepage}{\otp\addtocounter{framenumber}{-1}}

\begin{document}

\frame[plain]{\titlepage}

\begin{frame}{Analysis of Parallel C++ Programs}

\textbf{Analysis of Parallel Programs}
\begin{itemize}
    \item parallel processing inherent in modern hardware
    \item it is easy to make mistakes in parallel software
    \item problems hard to find by conventional testing
\end{itemize}
\pause

\bigskip
\textbf{C++}
\begin{itemize}
  \item often used in performance-critical applications
  \item high-level language
  \item has many features present in other languages (and not present in C)
    \pause
    \begin{itemize}
      \item objects, inheritance, threading, exceptions, …
    \end{itemize}
\end{itemize}
\end{frame}

\begin{frame}{TODO}
  - DIVINE
\end{frame}

\begin{prespart}{Improvements in Analysis of Realistic Programs}
  \begin{itemize}
    \item \fcite{SRB2017}
  \end{itemize}
\end{prespart}

\begin{frame}{Component Reuse}
  analysis of high-level programming languages is complex
  \begin{itemize}
    \item syntax (\cpp{for ( auto [ k, v ] : map ) { ...  }})
    \item code-generation (\cpp{template< typename T > class vector { ... }})
    \item object-based polymorphism, run-time types, virtual methods
    \item exceptions
    \item libraries
    \item … \pause
    \item → expensive to implement from scratch
  \end{itemize}

  \pause\bigskip
  we reuse existing components
  \begin{itemize}
    \item compiler (clang)
    \item some libraries (pdclib, libc++)
  \end{itemize}

  and implement other
  \begin{itemize}
    \item parts of exception handling (libunwind)
    \item threads (pthread)
  \end{itemize}
\end{frame}

\begin{frame}[fragile]{Exceptions in C++}
  \begin{minipage}{0.47\textwidth}
    \setlength{\leftmargini}{1em}
    \begin{itemize}
      \item ubiquitous in C++
      \item disabling them can change behaviour (e.g. allocation)
      \item need complex runtime support
        \begin{itemize}
          \item exception matching
          \item cleanup
          \item stack unwinding
        \end{itemize}
    \end{itemize}
  \end{minipage}
  \hfill
  \pause
  \newcommand{\arr}[1]{\visible<#1>{\textcolor{red}{$\mathbf{\longleftarrow}$}}}
  \begin{minipage}{0.47\textwidth}
    \begin{cppcodeln}
      X::~X() { } ⟦\arr{7}⟧

      void g() {
        throw std::out_of_range(); ⟦\arr{5}⟧
      }
      void f() {
        X x; ⟦\arr{3}⟧
        g(); ⟦\arr{4}⟧
      } ⟦\arr{6}⟧
      int main() {
        try {
          f(); ⟦\arr{2}⟧
        } catch ( std::logic_error ) {
          /* ... */ ⟦\arr{8}⟧
        }
      }
    \end{cppcodeln}
  \end{minipage}
\end{frame}

\begin{frame}{Exceptions in DIVINE}
  \textbf{Goals}
  \begin{itemize}
    \item reuse as much of the C++ library implementation as possible
    \item limit changes to the core of DIVINE
    \item full support for C++ exceptions
  \end{itemize}
  \pause

  \bigskip
  \textbf{Our Approach}
  \begin{itemize}
    \item calculate exception handling metadata for the LLVM code
    \item unwinding library that uses them
    \item lower some exception handling primitives in LLVM to our functions
  \end{itemize}
  \pause

  \bigskip
  \textbf{Results}
  \begin{itemize}
    \item just 300 lines of code for metadata and 210 lines of code for unwinding
    \item comparable performance to (partial) implementation that requires substantial changes to
      the C++ library implementation and DIVINE
  \end{itemize}
\end{frame}

\begin{prespart}{Analysis of Programs Under the \xtso Relaxed Memory Model}
  \begin{itemize}
    \item \fcite{SB2018x86tso}
  \end{itemize}
\end{prespart}

\begin{frame}[fragile]{Relaxed Memory}
  \begin{minipage}{0.47\textwidth}
    modern processors use various techniques to speed up processing
    \begin{itemize}
      \item cache hierarchies
      \item instruction reordering
      \item speculative execution
    \end{itemize}
    \pause

    this gives rise to observable relaxed behaviour
  \end{minipage}
  \hfill
  \begin{minipage}{0.48\textwidth}
    \begin{cppcode}
      int x = 0;
      int y = 0;
    \end{cppcode}
    \begin{minipage}{0.48\textwidth}
      \begin{cppcode}
        void thread0()
        {
          y = 1;
          int a = x;
        }
      \end{cppcode}
    \end{minipage}
    \hfill
    \begin{minipage}{0.48\textwidth}
      \begin{cppcode}
        void thread1()
        {
          x = 1;
          int b = y;
        }
      \end{cppcode}
    \end{minipage}

    \centering\medskip
    Is $\texttt{a} = 0 \land \texttt{b} = 0$ reachable?
  \end{minipage}
\end{frame}

\begin{prespart}{Local Nontermination analysis for Parallel Programs}
  \begin{itemize}
    \item \fcite{SB2019}
  \end{itemize}
\end{prespart}

\begin{prespart}{Other Significant Publications}
  \begin{itemize}
    \item \fcite{DIVINEToolPaper2017}
    \item \fcite{RSCB2018}
  \end{itemize}
\end{prespart}

\begin{prespart}{Conclusion}
\end{prespart}

\end{document}

% vim: fo+=t tw=100 spell spelllang=en shiftwidth=2
