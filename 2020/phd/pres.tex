\documentclass[aspectratio=169, fi]{paradise-slide}

\setbeamertemplate{caption}[numbered]
\setbeamertemplate{caption label separator}{:}
\setbeamercolor{caption name}{fg=normal text.fg}
\usepackage{amssymb,amsmath}
\usepackage{ifxetex,ifluatex}
\usepackage{fixltx2e} % provides \textsubscript

\usepackage{upquote}
\usepackage{microtype}
\usepackage{booktabs}

\usetikzlibrary{arrows,shapes,snakes,automata,backgrounds,petri,fit}

% Comment these out if you don't want a slide with just the
% part/section/subsection/subsubsection title:
\AtBeginPart{
  \let\insertpartnumber\relax
  \let\partname\relax
  \frame{\partpage}
}
\AtBeginSection{
  \let\insertsectionnumber\relax
  \let\sectionname\relax
  \frame{\sectionpage}
}
\AtBeginSubsection{
  \let\insertsubsectionnumber\relax
  \let\subsectionname\relax
  \frame{\subsectionpage}
}

\setlength{\parindent}{0pt}
\setlength{\emergencystretch}{3em}  % prevent overfull lines
\providecommand{\tightlist}{%
  \setlength{\itemsep}{0pt}\setlength{\parskip}{0pt}}
\setcounter{secnumdepth}{0}
\ifxetex
  \usepackage{polyglossia}
  \setmainlanguage{}
  \setotherlanguages{}
\else
  \usepackage[shorthands=off,english]{babel}
\fi
\usepackage{minted}

\newminted[cppcode]{cpp}{autogobble,breaklines,escapeinside=⟦⟧}
\newminted[cppcodeln]{cpp}{autogobble,breaklines,linenos,escapeinside=⟦⟧}
\newmintinline[cpp]{cpp}{}

\usepackage[ backend=biber
           , style=alphabetic % numeric
           , sortlocale=en_US
           , bibencoding=UTF8
           , sorting=anyt % explicit label, name, year, title
           , sortlocale=cs_CZ
           , maxnames=100
           , maxalphanames=4
           , maxbibnames=100
           , urldate=long
           ]{biblatex}
\DeclareSourcemap{
    \maps[datatype=bibtex, overwrite]{
        \map{
            \step[fieldset=editor, null]
            \step[fieldset=language, null]
            \step[fieldset=url, null]
            \step[fieldset=isbn, null]
        }
    }
}
\addbibresource{my.bib}
\newcommand{\fcite}[1]{\emergencystretch 3em{\protect\NoHyper\cite{#1}}~\fullcite{#1}}

\usepackage{xspace}
\newcommand{\xtso}{\mbox{x86-\kern-.15em TSO}\xspace}

\title{Analysis of Parallel C++ Programs}
\subtitle{PhD Thesis Defense}
\author{Vladimír Štill}
\date{3rd December 2020}

\newenvironment{prespart}[1]{%
  \begin{frame}{}%
    \centering
      {\Large #1} \par\bigskip\bigskip%
}{%
  \end{frame}%
}
\newcommand{\rquote}[1]{\begin{quote}#1\end{quote}\bigskip\setlength{\leftmargini}{1em}}

\begin{document}

\frame[plain, noframenumbering]{\titlepage}

\begin{frame}{Analysis of Parallel C++ Programs}

\textbf{Automatic Analysis of Parallel Programs}
\begin{itemize}
    \item parallel processing to make use of modern hardware
    \item easy to make mistakes in parallel software
    \item problems hard to find by conventional testing
\end{itemize}
\pause

\bigskip
\textbf{C++}
\begin{itemize}
  \item often used in performance-critical applications
  \item high-level language
  \item has many features present in other languages (and not present in C)
    \pause
    \begin{itemize}
      \item complex objects, inheritance, threading, atomic variables, exceptions, …
    \end{itemize}
\end{itemize}
\end{frame}

\begin{frame}{DIVINE}
  All work presented in the thesis is implemented and evaluated in DIVINE

  \begin{itemize}
    \item explicit-state analyser for C and C++, using LLVM intermediate representation
      \begin{enumerate}
        \item C/C++ translated to LLVM using clang compiler
        \item LLVM is instrumented
        \item and executed by DIVINE in search for errors
      \end{enumerate}\pause
    \item primarily targets reachability, also supports Büchy and termination conditions
    \item symbolic extensions \pause
    \item multiple other contributors, parts presented here authored mostly by me
    \item open source
  \end{itemize}
\end{frame}

\begin{prespart}{Improvements in Analysis of Realistic Programs}
  \begin{itemize}
    \item \fcite{SRB2017} \hfill 75\,\%
  \end{itemize}
\end{prespart}

\begin{frame}{Component Reuse}
  analysis of high-level programming languages is complex \pause
  \begin{itemize}
    \item syntax (\cpp{for ( auto [ k, v ] : map ) { ...  }}) \pause
    \item code-generation (\cpp{template< typename T > class vector { ... }}) \pause
    \item object-based polymorphism, run-time type support, virtual methods \pause
    \item exceptions \pause
    \item libraries
    \item … \pause
    \item → expensive to implement from scratch
  \end{itemize}

  \pause\bigskip
  we reuse existing components
  \begin{itemize}
    \item compiler (clang)
    \item some libraries (pdclib, libc++)
  \end{itemize}

  and implement other
  \begin{itemize}
    \item threads (pthread)
    \item parts of exception handling (libunwind)
  \end{itemize}
\end{frame}

\newcommand{\arr}[1]{\visible<#1>{\textcolor{red}{$\mathbf{\longleftarrow}$}}}
\begin{frame}[fragile]{Exceptions in C++}
  \begin{minipage}{0.47\textwidth}
    \setlength{\leftmargini}{1em}
    \begin{itemize}
      \item ubiquitous in C++
      \item disabling them can change behaviour (e.g. allocation)
      \item need complex runtime support
        \begin{itemize}
          \item exception matching
          \item cleanup
          \item {\only<2->\bf stack unwinding}
        \end{itemize}
    \end{itemize}

    \bigskip
    \visible<2->{Stack:}\medskip\\
    \visible<5,7>{\framebox[0.8\textwidth]{\vphantom{gf}\only<5>{\cpp{g}}\only<7>{\cpp{X::~X}}}}
    \visible<3-7>{\framebox[0.8\textwidth]{\vphantom{gf}\cpp{f}}}
    \visible<2->{\framebox[0.8\textwidth]{\vphantom{gf}\cpp{main}}}
  \end{minipage}
  \hfill
  \pause
  \begin{minipage}{0.47\textwidth}
    \begin{cppcodeln}
      X::~X() { } ⟦\arr{7}⟧

      void g() {
        throw std::out_of_range(); ⟦\arr{5}⟧
      }
      void f() {
        X x; ⟦\arr{3}⟧
        g(); ⟦\arr{4}⟧
      } ⟦\arr{6}⟧
      int main() {
        try {
          f(); ⟦\arr{2}⟧
        } catch ( std::logic_error ) {
          /* ... */ ⟦\arr{8}⟧
        }
      }
    \end{cppcodeln}
  \end{minipage}
\end{frame}

\begin{frame}{Exceptions in DIVINE}
  \textbf{Goals}
  \begin{itemize}
    \item reuse as much of the C++ library implementation as possible (libc++abi)
    \item limit changes to the core of DIVINE
    \item full support for C++ exceptions
  \end{itemize}
  \pause

  \bigskip
  \textbf{Our Approach}
  \begin{itemize}
    \item calculate exception handling metadata for the LLVM code in DIVINE
    \item implement unwinding library using this metadata
    \item lower some exception handling primitives in LLVM to our functions
  \end{itemize}
  \pause

  \bigskip
  \textbf{Results}
  \begin{itemize}
    \item no modifications to libc++abi required
    \item comparable performance to (partial) implementation that requires substantial changes to
      the C++ library implementation and DIVINE
    \item makes use of the new state representation in DIVINE 4
  \end{itemize}
\end{frame}

\begin{prespart}{Analysis of Programs Under the \xtso Relaxed Memory Model}
  \begin{itemize}
    \item \fcite{SB2018x86tso} \hfill 90\,\%
  \end{itemize}
\end{prespart}

\begin{frame}[fragile]{Relaxed Memory}
  \begin{minipage}{0.47\textwidth}
    modern processors use various techniques to speed up processing
    \begin{itemize}
      \item cache hierarchies
      \item instruction reordering
      \item speculative execution
    \end{itemize}
    \pause

    this gives rise to observable relaxed behaviour
  \end{minipage}
  \hfill\pause
  \begin{minipage}{0.48\textwidth}
    \begin{cppcode}
      int x = 0;
      int y = 0;
    \end{cppcode}
    \begin{minipage}{0.49\textwidth}
      \begin{cppcode}
        void thread0()
        { ⟦\arr{6}⟧
          y = 1; ⟦\arr{7}⟧
          int a = x; ⟦\arr{8-}⟧
        }
      \end{cppcode}
    \end{minipage}
    \hfill
    \begin{minipage}{0.49\textwidth}
      \begin{cppcode}
        void thread1()
        { ⟦\arr{6-8}⟧
          x = 1; ⟦\arr{9}⟧
          int b = y; ⟦\arr{10}⟧
        }
      \end{cppcode}
    \end{minipage}

    \centering\medskip
    Is $\texttt{a} = 0 \land \texttt{b} = 0$ reachable?
  \end{minipage}

  \bigskip\bigskip\pause
  \begin{minipage}[t]{0.48\textwidth}
    Not under \emph{Sequential Consistency}

    \visible<6->{
      \medskip
      Globals: \texttt{x = \only<-8>{0}\only<9->{1}}, \texttt{y = \only<-6>{0}\only<7->{1}}\\
      Locals: \texttt{a = \only<-7>{$\bot$}\only<8->{0}}, \texttt{b = \only<-9>{$\bot$}\only<10->{1}}\\
    }
  \end{minipage}
  \hfill\pause
  \begin{minipage}[t]{0.48\textwidth}
    Reachable on x86, ARM, POWER, …

    \visible<6->{
      \medskip
      Globals: \texttt{x = \only<-10>{0}\only<11->{1}}, \texttt{y = 0}\\
      Locals: \texttt{a = \only<-7>{$\bot$}\only<8->{0}}, \texttt{b =
        \only<-9>{$\bot$}\only<10->{\textcolor{red}{0}}}\\
      Store buffer t0: \textcolor{red}{\only<7->{y = 1}} \\
      Store buffer t1: \textcolor{red}{\only<9-10>{x = 1}}  \\
    }
  \end{minipage}
\end{frame}

\begin{frame}{Analysis of Programs Under the x86-TSO Relaxed Memory Model}
  Preexisting approaches
  \begin{itemize}
    \item stateless model checking (Nidhugg, RCMC, …)
    \item bounded model checking (CBMC, …)
    \item explicit-state model checking (not very competitive)
    \item also techniques which consider any relaxed behaviour as error
  \end{itemize}

  \bigskip
  Our approach
  \begin{itemize}
    \item \xtso – memory model of Intel and AMD x86(-64) CPUs
    \item C and C++
    \item based on explicit-state model checking
  \end{itemize}
\end{frame}

\begin{frame}[fragile]{Efficient explicit simulation of \xtso}
  \begin{itemize}
    \item \xtso allows arbitrary delaying of stores
    \item only some possibilites observable
    \item flushing store buffers only if someone reads buffered values
    \item delayed flushing – some values can behave as flushed, but their order respective to other
      buffered values can be undetermined
  \end{itemize}

  \pause\bigskip
  \colorlet{frombuf}{green!60!black}
  \colorlet{flushed}{yellow!60!red!85!black}
  \begin{minipage}[t]{0.4\textwidth}

  \begin{tikzpicture}[ ->, >=stealth', shorten >=1pt, auto, node distance=3cm
                     , semithick
                     , scale=0.5
                     , thck/.style = { thick, decoration={markings,mark=at position 1 with {\arrow[scale=4]{>}}}, postaction={decorate}, },
                     ]

    \draw [-] (-10,0) rectangle (-7,-4);
    \draw [-] (-10,-1) -- (-7,-1)
              (-10,-2) -- (-7,-2)
              (-10,-3) -- (-7,-3);
    \node () [anchor=center] at (-8.5, 0.5) {s.b. 1};
    \node () [anchor=center] at (-8.5,-0.5) {\only<-5>{\only<4->{\color{flushed}}\texttt{x $\leftarrow$ 1}}};
    \node () [anchor=center] at (-8.5,-1.5) {\only<-3>{\texttt{y $\leftarrow$ 1}}\only<4-5>{\color{flushed}\texttt{x $\leftarrow$ 2}}};
    \node () [anchor=center] at (-8.5,-2.5) {\only<-3>{\texttt{x $\leftarrow$ 2}}};
    \node () [anchor=center] at (-8.5,-3.5) {\only<-3>{\only<3>{\color{frombuf}}\texttt{y $\leftarrow$ 2}}};

    \draw [-] (-6,0) rectangle (-3,-4);
    \draw [-] (-6,-1) -- (-3,-1)
              (-6,-2) -- (-3,-2)
              (-6,-3) -- (-3,-3);
    \node () [anchor=center] at (-4.5, 0.5) {s.b. 2};
    \node () [anchor=center] at (-4.5,-0.5) {\only<-6>{\only<5-6>{\color{frombuf}}\texttt{x $\leftarrow$ 3}}\only<7->{\texttt{y $\leftarrow$ 3}}};
    \node () [anchor=center] at (-4.5,-1.5) {\only<-6>{\texttt{y $\leftarrow$ 3}}};

  \end{tikzpicture}

  \newcommand{\ly}{\only<3->{\only<3>{\textcolor{frombuf}}{// \textrightarrow 3}}}
  \newcommand{\lx}{\only<5->{\only<5-6>{\textcolor{frombuf}}{// \textrightarrow 4}}}
  \begin{cppcode}
  void thread0() {
    int a = y; ⟦\ly⟧
    int b = x; ⟦\lx⟧
  }
  \end{cppcode}
  \end{minipage}
  \begin{minipage}[t]{0.55\textwidth}
      \begin{itemize}
          \only<1>{\item}
          \only<2>{\item three threads, thread 0 executing}
          \only<3>{\item nondeterministically choose which value to load from the
          store buffers}
          \only<4>{\item older values for \texttt{y} in s.b. 1 removed}
          \only<4>{\item other older values in s.b. 1 marked as flushed}
          \only<5>{\item nondeterministically choose to load value from s.b.~2}
          \only<6>{\item flushed values for \texttt{x} has to be removed (overwritten)}
          \only<7>{\item \texttt{x} is propagated to the memory}
          \only<8>{\item \texttt{y $\leftarrow$ 3} stays in buffer unless \texttt{y} is read again}
      \end{itemize}
  \end{minipage}
\end{frame}

\begin{frame}{Results}

  \begin{itemize}
    \item evaluation on nonsymbolic SV-COMP concurrency benchmarks
    \item compared with CBMC and Nidhugg

    \vspace{2ex}
    \setlength\tabcolsep{0.5em} %
    \begin{tabular}{lrrr} \toprule
        \textcolor{gray}{(total = 54)} & CBMC & \textbf{DIVINE} & Nidhugg \\ \midrule
        finished                & 20   & 23     & 24 \\
        TSO errors              &  3   & 10     &  3 \\
        uniqly solved           &  5   &  6     &  8 \\
        \bottomrule
    \end{tabular}
  \end{itemize}
\end{frame}

\begin{prespart}{Local Nontermination analysis for Parallel Programs}
  \begin{itemize}
    \item \fcite{SB2019} \hfill 90\,\%
  \end{itemize}
\end{prespart}

\begin{frame}{Beyond Safety Checking}
  safety checking is common for realistic parallel programs
  \begin{itemize}
    \item assertion violations, memory corruption, data races…
    \item caused by thread interleaving, possibly with relaxed memory
    \pause
    \item if the program does not terminate…
    \begin{itemize}
      \item the tool might not terminate
      \item or it might report there are no safety violations (correctly)
    \end{itemize}
  \end{itemize}

  \pause\bigskip
  → often we need to go beyond safety checking
  \begin{itemize}
    \item termination
    \item temporal properties (liner, branching time, …)
  \end{itemize}
\end{frame}

\begin{frame}[fragile]{Local Nontermination analysis for Parallel Programs}
  \begin{minipage}{0.49\textwidth}
    checking that some patrs of the program terminate
    \begin{itemize}
      \item critical sections
      \item waiting for entry to critical sections
      \item waiting for condition variables, thread termination
      \item user-defined parts
    \end{itemize}
  \end{minipage}
  \hfill
  \begin{minipage}{0.47\textwidth}
    \begin{cppcode}
      void foo() {
        // lock
        std::unique_lock guard( mtx );
        do_work()
        cond.notify_all();
      } // unlock
    \end{cppcode}
  \end{minipage}
\end{frame}

\begin{frame}[fragile]{Local Nontermination analysis for Parallel Programs}
  \setlength{\leftmargini}{1em}
  \begin{itemize}
    \item we aim at nontermination caused by unintended parallel interactions \pause
    \item not at complex control flow \& loops \pause
    \item should be easy to specify
    \item should not report nontermination spuriously
    \item should be useful for analysis of services/servers – programs that should not terminate
  \pause
  \bigskip
    \item builds on explicit-state model checking → finite-state programs (with possibly infinite behaviour)
    \item user can specify what to check
  \end{itemize}

  \bigskip
  \begin{minipage}{0.49\textwidth}
    \begin{cppcode}
      bool x = true;
      while (true) { x = !x; }
    \end{cppcode}
  \end{minipage}
  \hfill
  \begin{minipage}{0.49\textwidth}
    \begin{tikzpicture}[>=latex,>=stealth',auto,node distance=2cm,semithick,initial text=]
      \node[state,initial] (t) {$x$};
      \node[state] (f) [right of = t] {$\lnot x$};

      \path[->, shorten >=1pt]
        (t) edge[bend left] (f)
        (f) edge[bend left] (t)
      ;
    \end{tikzpicture}
  \end{minipage}
\end{frame}

\begin{frame}[fragile]{What is Nontermination?}
  \begin{cppcode}
  std::atomic< bool > spin_lock;
  void worker() { // two of these running in parallel
      while (true) {
          while (spin_lock.exchange(true)) { /* wait */ }
          x++;
          spin_lock = false;
      }
  }
  \end{cppcode}

  Does every \emph{wait} end? \pause

  \bigskip
  \newcommand{\ta}[1]{\text{\textcolor{blue}{\texttt{[t0: #1]}}}}
  \newcommand{\tb}[1]{\text{\textcolor{red}{\texttt{[t1: #1]}}}}
  $\ta{spin\_lock.exchange(true) \textrightarrow{} false}$\\
  $\big(\tb{spin\_lock.exchange(true) \textrightarrow{} true}$\\
  $\phantom{\big(}\ta{x++}$\\
  $\phantom{\big(}\ta{spin\_lock = false}$\\
  $\phantom{\big(}\ta{spin\_lock.exchange(true) \textrightarrow{} false}\big)^\omega$

  \medskip
  \textbf{both threads can run}
\end{frame}

\begin{frame}[fragile]{Local Nontermination analysis for Parallel Programs}
%  \begin{minipage}{0.66\textwidth}
    Resource Section
    \begin{itemize}
      \item a block of code with an identifier
      \item delimited in the source code
      \item built-in (mutexes, condition variables, thread joining, …)
      \item user-provided (in source code; block of code, function end, …)
    \end{itemize}
    \pause

    \bigskip
    Local Nontermination
    \begin{itemize}
      \item a resource section does not terminate if the program can reach apoint \emph{in the
        resource section} from which it cannot reach \emph{the corresponding resource section end}
    \end{itemize}
    \pause

    \bigskip
    Detection
    \begin{itemize}
      \item nondeterministically \emph{activate} resource sections at their beginning
      \item active resource sections are checked for nontermination, cannot be nested
      \item detect terminal strongly connected components in active resource sections
    \end{itemize}
%  \end{minipage}
%  \hfill
%  \begin{minipage}{0.33\textwidth}
%    \begin{tikzpicture}[>=latex,>=stealth',auto,node distance=1.5cm,semithick,initial
%                        text=, ->, shorten >=1pt, initial distance = 1cm]
%      \tikzstyle{state}=[circle, draw, minimum size=0.75cm]
%      \tikzstyle{scc}=[draw, dashed, inner sep = 0.3cm, rounded corners=0.4em]
%      \node[state,initial] (c1a) {};
%      \node[state] (c1b) [right of = c1a] {};
%      \node[state] (c1c) [below of = c1a] {};
%
%      \path
%        (c1a) edge[bend left] (c1b)
%        (c1b) edge[bend left] (c1c)
%        (c1c) edge[bend left] (c1a)
%        (c1b) edge[bend left] (c1a)
%      ;
%
%      \node[scc, fit = (c1a) (c1b) (c1c)] (c1) {};
%
%      \node[state] (c2a) [below of = c1c] {};
%      \path (c1c) edge (c2a);
%      \node[scc, fit=(c2a)] (c2) {};
%
%      \node[state] (c4a) [below of = c2a] {};
%      \node[state] (c4b) [below of = c4a] {};
%      \path
%        (c2a) edge (c4a)
%        (c4a) edge[bend left] (c4b)
%        (c4b) edge[bend left] (c4a)
%      ;
%      \node[scc, fit=(c4a) (c4b)] (c4) {};
%      % \node[anchor=south, rotate = -90, yshift = -0.5cm] at (c4.west) {nontriv. terminal SCC};
%
%      \node[state] (c5a) [right of = c4a, node distance=2cm] {};
%      \node[state] (c5b) [below of = c5a] {};
%      \path
%        (c2a) edge[bend left] (c5a)
%        (c5a) edge[bend left, line width=3pt, draw=blue] (c5b)
%        (c5b) edge[bend left, line width=3pt, draw=blue] (c5a)
%      ;
%      \node[scc, fit=(c5a) (c5b), line width=3pt, draw=blue] (c5) {};
%      \node[anchor=south, rotate = -90] at (c5.east) {FATSCC};
%    \end{tikzpicture}
%  \end{minipage}
\end{frame}

\begin{prespart}{Other Significant Publications}
  \begin{itemize}
    \item \fcite{DIVINEToolPaper2017} \hfill 30\,\%
    \bigskip
    \item \fcite{RSCB2018} \hfill 20\,\%
  \end{itemize}
\end{prespart}

\begin{prespart}{Conclusion}
  \begin{itemize}
    \item contributions to analysis of parallel programs in C++
      \begin{itemize}
        \item realistic programs, compiler and standard library reuse, and exceptions
          \\ – \emph{new approach to implementation of exceptions}
        \item programs running under the \xtso memory model
          \\ – \emph{new efficient way to simulate \xtso run in an explicit-state model checker}
        \item local nontermination detection
          \\ – \emph{new technique that can detect that a part of the program which is supposed to
          terminate does not}
      \end{itemize}
    \item all implemented in DIVINE and evaluated
  \end{itemize}
\end{prespart}

\def\insertframenumber{A}
\begin{frame}[noframenumbering]{}
    \centering
      {\Large Reactions to Reviews}
\end{frame}

\def\rname{Tomáš Vojnar}
\def\qtitle{Questions by \rname}

\begin{frame}[fragile, noframenumbering]{\qtitle}
\rquote{What I am slightly missing in the chapter [4] is a better explanation of which particular features
of exception handling were not supported by DIVINE 3.}

  \begin{itemize}
    \item exception specifications
      \begin{cppcode}
        void bar(int x) throw() { ... }
        void foo(int x) throw (std::out_of_range) { ... }
      \end{cppcode}
    \item new approach also allows for implementation of \cpp{longjmp}/\cpp{setjmp} for the sake of
      C~programs
    \item DIVINE 3 does not check if the target function is using C++ exceptions
  \end{itemize}
\end{frame}

\begin{frame}[noframenumbering]{\qtitle}
  Handling of unaligned loads in \xtso.\bigskip

  \begin{itemize}
    \item alignment is not significant for our \xtso simulation, as long as the there are no
      \emph{partially overlapping} loads/stores
      \begin{itemize}
        \item this matches with the observed behaviour of x86-64 processors and the \xtso
          theoretical memory model
      \end{itemize}
    \item partial overlaps should not happen for atomic variables in C++ (undefined behaviour)
    \item but can happen for non-atomics (e.g. \cpp{memcpy}) TODO
  \end{itemize}
\end{frame}

\begin{frame}[fragile, noframenumbering]{\qtitle}
  \rquote{Why do you concentrate solely on instances of resource sections that are bound to
    non-terminate? Why do you not consider these that can but need not terminate too? I understand
    that such non-termination can happen under weak fairness when accessing shared resources, but
    why not considering strong fairness (at least wrt access to some resources) and non-termation
    that is possible (though not necessary) under such fairness too?}

  Consider this example from a single-prodcer single-consumer queue using atomics:
  \begin{cppcode}
    T &front() {
        while ( empty() ) { /* wait */ }
        // ...
  \end{cppcode}
  \begin{itemize}
    \item empty is a non-blocking operation, therefore the thread is always enabled at this line
    \item we tried to avoid false nontermination errors in common synchronization patterns
    \item it would be possible to have also sections that check for other fairness criteria
  \end{itemize}
\end{frame}

\begin{frame}[noframenumbering]{\qtitle}
  \rquote{Where are the predefined resource sections applied in DIVINE? Is this within
    implementation of various synchronization primitives, or are you trying to detect even
    application-specific resource sections automatically?}

  \begin{itemize}
    \item in synchronization primitives defined in DIVINE (mostly in the \texttt{pthread} library)
    \item automatic detection might be possible for wait-style sections, but would be hard for
      mutual exclusion
    \item with automatic detection, there is a risk that a small change in source (or compiler)
      would interfere with detection
  \end{itemize}
\end{frame}

\def\rname{Jaco van de Pol}

\begin{frame}[noframenumbering]{\qtitle}
  \rquote{Could you summarize the main problem(s) solved by this thesis, but \emph{not} in related
    work?}
  \begin{itemize}
    \item 
  \end{itemize}
\end{frame}

\begin{frame}[noframenumbering]{\qtitle}
  \rquote{Could you summarize the main new technique(s) developed by you in this thesis?}
  \begin{itemize}
    \item 
  \end{itemize}
\end{frame}

\begin{frame}[noframenumbering]{\qtitle}
  \rquote{You state that the verification of nearly all features of C++ and its standard library are
    supported. Which features are not supported, why not, and how fundamental are they?}

  most of C++17 is supported, missing features include:
  \begin{itemize}
    \item some functionality related to locale support (due to missing support in our libc
      implementation)
      \begin{itemize}
        \item unlikely to be used in parallel algorithms and data structures
      \end{itemize}
    \item filesystem support is limited by abilities of virtual file system in DIVINE
    \item parts of \texttt{math.h} (mostly related to loating-point environment)
    \item handling of real time
    \item networking (defined in platform APIs, not C++)
  \end{itemize}
\end{frame}

\begin{frame}[noframenumbering]{\qtitle}
  \rquote{You transform an ``LLVM program under TSO'' to an ``LLVM program under SC''. What
  scientific support do you have that the transformation preserves semnatics? What properties of the
  program are actually preserved by the transformation? (e.g. all linerar-time properties? only
  stutter-free properties? fair traces? termination? branching properties?)}
  \begin{itemize}
    \item the argumentation stated in the paper and a suite of test which test reachable results of
      thread interactions
    \item we considered mostly reachability
      \begin{itemize}
        \item automata-based approach to LTL would likely lead to infinite state space with store
          buffer limits, or unsound analysis without them
        \item as long as values used to evaluate the formula (or syncrhonize Büchy automaton, etc)
          are read using the proposed transformation, unbounded state space should preserve
          stutter-free linear properties
      \end{itemize}
    \item DIVINE can reliably analyse only stutter-free linerar-time properties and has no support
      for branching properties
  \end{itemize}
\end{frame}

\begin{frame}[noframenumbering]{\qtitle}
  \rquote{You compare your solution to weak-fairness, but how does your solution compare to strong
    fairness? Why are the fairness assumptions you postulate justified in reality?}
  \begin{itemize}
    \item strong fairness is not strong enough to eliminate spurious conterexamples in common
      synchronization patterns used with atomic variable
    \item 
  \end{itemize}
\end{frame}

\begin{frame}[noframenumbering]{\qtitle}
  \rquote{Can it happen that the tool concludes termination, just because a malloc-function returns
    NULL (out-of-memory) non-deterministically?}
  \begin{itemize}
    \item yes (the problem is same as for symbolic data)
  \end{itemize}
\end{frame}

\begin{frame}[noframenumbering]{\qtitle}
  \rquote{Do you have concrete ideas on how to integrate your advances with more symbolic or more
    modular apporaches?}

  Symbolic approaches
  \begin{itemize}
    \item DIVINE has support for compiled data abstractions, both precise (bitvectors) and lossy
    \item exception support does not interfere with symbolic approaches (as long as control-flow is
      explicit)
    \item conceptually, \xtso simulation does not intefere with symbolic approaches
      \begin{itemize}
        \item implementation would be tricky in DIVINE
        \item the symbolic extension relies on over-approximation of set of possibly symbolic
          locations this would likely encompass the whole program if any symbolic value was stored
          using store buffers
      \end{itemize}
    \item local nontermination would need a modified fairness condition to handle nondeterminism
      without under-approximation
  \end{itemize}
\end{frame}

\begin{frame}[noframenumbering]{\qtitle}
  \rquote{Do you have concrete ideas on how to integrate your advances with more symbolic or more
    modular apporaches?}

  Modular approaches
  \begin{itemize}
    \item I have rather limited knowledge of modular approaches to software verification
    \item exceptions
      \begin{itemize}
        \item symbolic analysis could be used to find conditions under which exception can
          or must be raised by a function
        \item however, our apporoach to exceptions is badly suited for axiomatic analysis usual for
          modular methods
      \end{itemize}
    \item \xtso – TODO
    \item local nontermination – thread-modular apporoaches seem to be orthogonal to our approach,
      it might be possible to use them to limit the number of resource sections that need to be
      checked
  \end{itemize}
\end{frame}

\begin{frame}[noframenumbering]{\qtitle}
  From text comments:
  \rquote{[Chapter 4] Is the solution sound with respect to the (intended) semantics of exceptions? In
    particular, what about corner cases, like an exception is raised during handling another
    exception (maybe even during stack unwinding/cleanup handlers).}

  yes
  \begin{itemize}
    \item these cases are handled by the part of C++ library that is re-used, so the behaviour
      shoudl be equivalent to the behaviour presented in normal execution
    \item only difference could be in parallel access to exception by the undwinder and some other
      thread, however:
      \begin{itemize}
        \item the exception object is thread-local
        \item while it contains user payload which might be globally visible, it does not access it
          (it does not have any knowledge of its structure)
      \end{itemize}
  \end{itemize}
\end{frame}


\end{document}

% vim: fo+=t tw=100 spell spelllang=en shiftwidth=2
